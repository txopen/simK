% Options for packages loaded elsewhere
\PassOptionsToPackage{unicode}{hyperref}
\PassOptionsToPackage{hyphens}{url}
%
\documentclass[
]{article}
\title{simK: Synthetic Data on Kidney Transplantation}
\author{}
\date{\vspace{-2.5em}02 August 2023}

\usepackage{amsmath,amssymb}
\usepackage{lmodern}
\usepackage{iftex}
\ifPDFTeX
  \usepackage[T1]{fontenc}
  \usepackage[utf8]{inputenc}
  \usepackage{textcomp} % provide euro and other symbols
\else % if luatex or xetex
  \usepackage{unicode-math}
  \defaultfontfeatures{Scale=MatchLowercase}
  \defaultfontfeatures[\rmfamily]{Ligatures=TeX,Scale=1}
\fi
% Use upquote if available, for straight quotes in verbatim environments
\IfFileExists{upquote.sty}{\usepackage{upquote}}{}
\IfFileExists{microtype.sty}{% use microtype if available
  \usepackage[]{microtype}
  \UseMicrotypeSet[protrusion]{basicmath} % disable protrusion for tt fonts
}{}
\makeatletter
\@ifundefined{KOMAClassName}{% if non-KOMA class
  \IfFileExists{parskip.sty}{%
    \usepackage{parskip}
  }{% else
    \setlength{\parindent}{0pt}
    \setlength{\parskip}{6pt plus 2pt minus 1pt}}
}{% if KOMA class
  \KOMAoptions{parskip=half}}
\makeatother
\usepackage{xcolor}
\IfFileExists{xurl.sty}{\usepackage{xurl}}{} % add URL line breaks if available
\IfFileExists{bookmark.sty}{\usepackage{bookmark}}{\usepackage{hyperref}}
\hypersetup{
  pdftitle={simK: Synthetic Data on Kidney Transplantation},
  hidelinks,
  pdfcreator={LaTeX via pandoc}}
\urlstyle{same} % disable monospaced font for URLs
\usepackage[margin=1in]{geometry}
\usepackage{graphicx}
\makeatletter
\def\maxwidth{\ifdim\Gin@nat@width>\linewidth\linewidth\else\Gin@nat@width\fi}
\def\maxheight{\ifdim\Gin@nat@height>\textheight\textheight\else\Gin@nat@height\fi}
\makeatother
% Scale images if necessary, so that they will not overflow the page
% margins by default, and it is still possible to overwrite the defaults
% using explicit options in \includegraphics[width, height, ...]{}
\setkeys{Gin}{width=\maxwidth,height=\maxheight,keepaspectratio}
% Set default figure placement to htbp
\makeatletter
\def\fps@figure{htbp}
\makeatother
\setlength{\emergencystretch}{3em} % prevent overfull lines
\providecommand{\tightlist}{%
  \setlength{\itemsep}{0pt}\setlength{\parskip}{0pt}}
\setcounter{secnumdepth}{-\maxdimen} % remove section numbering
\newlength{\cslhangindent}
\setlength{\cslhangindent}{1.5em}
\newlength{\csllabelwidth}
\setlength{\csllabelwidth}{3em}
\newlength{\cslentryspacingunit} % times entry-spacing
\setlength{\cslentryspacingunit}{\parskip}
\newenvironment{CSLReferences}[2] % #1 hanging-ident, #2 entry spacing
 {% don't indent paragraphs
  \setlength{\parindent}{0pt}
  % turn on hanging indent if param 1 is 1
  \ifodd #1
  \let\oldpar\par
  \def\par{\hangindent=\cslhangindent\oldpar}
  \fi
  % set entry spacing
  \setlength{\parskip}{#2\cslentryspacingunit}
 }%
 {}
\usepackage{calc}
\newcommand{\CSLBlock}[1]{#1\hfill\break}
\newcommand{\CSLLeftMargin}[1]{\parbox[t]{\csllabelwidth}{#1}}
\newcommand{\CSLRightInline}[1]{\parbox[t]{\linewidth - \csllabelwidth}{#1}\break}
\newcommand{\CSLIndent}[1]{\hspace{\cslhangindent}#1}
\ifLuaTeX
  \usepackage{selnolig}  % disable illegal ligatures
\fi

\begin{document}
\maketitle

\hypertarget{summary}{%
\section{Summary}\label{summary}}

Synthetic data can be used to validate mathematical models and to train
machine learning algorithms. Furthermore, distinct subject-specific
investigations would necessitate different data. Privacy issues, data
sensitivity, copyrights, and legal constraints, on the other hand, are
frequently substantial impediments to accessing the necessary data. In
the case of kidney transplantation, different countries have their own
kidney allocation systems (KAS) in place, to balance the concepts of
fairness and utility in the distribution of such scarce resource
(\protect\hyperlink{ref-Lima:2020}{B. Lima and Alves 2020}). Access to
health data is becoming increasingly restricted, however synthetic data
can always be used to evaluate alternative KAS and its modifications.

The goal of this package is to allow generate synthetic data that can be
used on the evaluation and assessment of KAS in transplantation.

\hypertarget{statement-of-need}{%
\section{Statement of need}\label{statement-of-need}}

\texttt{simK} is an R (\protect\hyperlink{ref-R}{R Core Team 2021})
package that assembles functions that allow to generate synthetic data
sets with demographic, clinical and immunologic characteristics of wait
listed candidates for kidney transplantation and a pool of deceased
donors. Data generated with \texttt{simK} are particularly useful with
the package \texttt{histoc} (\protect\hyperlink{ref-Lima:2022a}{B. Lima
and Reis 2022}) and the application Kidney Allocation Rules Simulator
(KARS) (\protect\hyperlink{ref-Lima:2022b}{B. Lima, Henriques, and Alves
2022}). With this two packages it is possible to simulate allocation
rules implemented in Portugal (\protect\hyperlink{ref-PT}{Ministry of
Health 2007}), in countries within Eurotransplant
(\protect\hyperlink{ref-ET}{EuroTransplant 2020}), in the United Kingdom
(\protect\hyperlink{ref-UK}{National Health Service Blood and Transplant
2017}), and a system suggested by (\protect\hyperlink{ref-Lima:2013b}{B.
Lima, Mendes, and Alves 2013}).

Access to health data is generally limited by issues of confidentiality
of individuals' information. Researchers' access to this type of data is
usually dependent on consents for use, protocol approval, ethical review
approval, bureaucracies for data requests, and financial costs
(\protect\hyperlink{ref-Gonzales:2023}{Gonzales, Guruswamy, and Smith
2023}). Nevertheless, sometimes, when it is not possible to have access
to real data for proprietary or even logistical reasons, there is the
alternative of using synthetic data. This type of data can be created,
used, shared and discarded without major associated costs and is useful
for researchers, innovators and data entrepreneurs alike, however
different their purposes may be. Moreover, this data can be made
available and used, publicly and free of charge, by industry,
innovators, researchers, education and without legal, privacy, security,
financial, and intellectual property limitations
(\protect\hyperlink{ref-Walonoski:2018}{Walonoski et al. 2018}).

The US Census Bureau defines synthetic data as:: ``microdata records
created by statistically modeling original data and then using those
models to generate new data values that reproduce the original data's
statistical properties'' (\protect\hyperlink{ref-Philpott:2017}{Philpott
2017}). Likewise, synthetic data can thus be a substitute for real
records, useful for secondary uses that require realistic but not
necessarily real data (\protect\hyperlink{ref-Chen:2019}{Chen et al.
2019}).

Synthetic data generation is increasingly used in research to test and
evaluate the applicability of new statistical methods or new algorithms.
In such cases, data synthesis seeks to provide test data that can
reflect as accurately as possible the complexity of the real-world data
where an algorithm under evaluation is to be applied. Moreover,
synthetic data sets can also complement real data allowing researchers
to increase their sample size or add variables that are not present in
an original data set, enabling new levels of validation and comparison
to test and develop new machine learning methods and algorithms
(\protect\hyperlink{ref-Gonzales:2023}{Gonzales, Guruswamy, and Smith
2023}).

Several R (\protect\hyperlink{ref-R}{R Core Team 2021}) packages are
already available to synthesize different types of data
(\protect\hyperlink{ref-ctv:2021}{B. Lima 2021}). Although, none of
these can be used in the specific case of KAS simulation without having
as input your own data.

The \texttt{simK} outcomes are easily verifiable, modifiable, and
redefined, which is a clear benefit as a facilitator of transparency and
a desired continuous improvement.

\hypertarget{synthetic-data-sets}{%
\subsection{Synthetic data sets}\label{synthetic-data-sets}}

\hypertarget{donors-pool}{%
\subsubsection{Donors' pool}\label{donors-pool}}

A data frame with information for a pool of simulated donors can be
generated with the function \texttt{donors\_df()}.

For a given number of \texttt{n} rows, a data frame is generated with
columns:

\begin{itemize}
\tightlist
\item
  \emph{ID} unique identifier with the prefix `D';
\item
  \emph{bg} with the blood group generated from the parameter probs a
  vector with the probabilities for groups A, AB, B and O, respectively;
\item
  \emph{A1}, \emph{A2}, \emph{B1}, \emph{B2}, \emph{DR1}, \emph{DR2} HLA
  typing obtained according to \texttt{origin} option (with
  \texttt{replace\ =\ TRUE} we can generate a data frame without
  limitations on the number of rows);
\item
  age generated from a Normal distribution with \texttt{mean} and sd
  given by the user, values truncated by \texttt{lower} and
  \texttt{upper} boundaries;
\item
  \emph{DRI} when option \texttt{uk\ =\ TRUE}, Donor Risk Index is
  copmputed as described by \texttt{transplantr}
  (\protect\hyperlink{ref-transplantr}{Asher 2020})
\end{itemize}

HLA population origin has currently as valid options `PT' for Portuguese
(\protect\hyperlink{ref-Lima:2013a}{B. Lima and Alves 2013}), and
populations available from US National Marrow Donor Program
(\protect\hyperlink{ref-nmdp:2023}{NMDP 2023}):

\begin{itemize}
\tightlist
\item
  `API' - Asian / Pacific Islander
\item
  `AFA' - African American / Black
\item
  `CAU' - White / Caucasian
\item
  `HIS' - Hispanic
\end{itemize}

Defining \texttt{seed.number} allows for reproducibility.

\hypertarget{kidney-transplant-candidates}{%
\subsubsection{Kidney transplant
candidates}\label{kidney-transplant-candidates}}

A simulated waiting list for kidney transplant candidates, can be
generated with \texttt{candidates\_df()}.

For a given number of \texttt{n} rows, a data frame is generated with
columns:

\begin{itemize}
\tightlist
\item
  \emph{ID} unique identifier with the prefix `K';
\item
  \emph{bg} with the blood group generated from the parameter
  \texttt{probs.abo} a vector with the probabilities for groups A, AB, B
  and O, respectively (here by default, we assumed group O patients are
  more frequent);
\item
  \emph{A1}, \emph{A2}, \emph{B1}, \emph{B2}, \emph{DR1}, \emph{DR2} HLA
  typing obtained according to \texttt{origin} option (with
  \texttt{replace\ =\ TRUE} we can generate a data frame without
  limitations on the number of rows);
\item
  \emph{age} generated from a Normal distribution with \texttt{mean} and
  \texttt{sd} given by the user, values truncated by \texttt{lower} and
  \texttt{upper} boundaries;
\item
  \emph{dialysis} time on dialysis \textbf{in months}, values computed
  according to patients' blood group and hypersensitation status
  (\(cPRA > 85%
  \)): for patients with blood group O \textbf{and} Hypersinsitized time
  on dialysis obtained from \(N(85, 20)\); for those patients blood O
  \textbf{or} Hypersinsitized \(N(70,20)\); remaining patients have time
  on dialysis obtained from \(N(35,20)\);
\item
  \emph{cPRA} patients are classified in groups with probabilities given
  by \texttt{probs.cpra} for 0\%, 1\%-50\%, 51\%-85\% and 86\%-100\%,
  respectively. Within the groups \textgreater{} 0\%, cPRA are computed
  as random values from distributions \(P(\lambda = 30)\),
  \(P(\lambda = 70)\) and \(P(\lambda = 90)\);
\item
  \emph{Tier} patients are classified in two Tiers as described on
  POL186/11 -- Kidney Transplantation: Deceased Donor Organ Allocation
  from UK transplant (\protect\hyperlink{ref-UK}{National Health Service
  Blood and Transplant 2017}). In Tier A are patients with \(MS = 10\)
  or \(cPRA = 100%
  \) or \(time on dialysis > 7 years\), all remaing patients are
  classified as Tier B;
\item
  \emph{MS} matchabilily score are the deciles obtained from the number
  of donors on dataset \texttt{D10K} that are a match to each transplant
  candidate. This score takes into account a patient's blood type, HLA
  type and cPRA value. A patient with a MS = 1 is defined as easy to
  match and a MS = 10 as difficult to match
  (\protect\hyperlink{ref-UK}{National Health Service Blood and
  Transplant 2017}).
\item
  \emph{RRI} when option \texttt{uk\ =\ TRUE}, Recipient Risk Index is
  copmputed as described by \texttt{transplantr}
  (\protect\hyperlink{ref-transplantr}{Asher 2020}). To compute
  \emph{RRI}, variables \emph{age}, time on \emph{dialysis} (in days)
  and the probability of being \emph{diabetic} (obtained from
  \texttt{prob.dm}) are used. Also, we assumed all patients were on
  dialysis at time of listing.
\item
  \emph{urgent} a diccotomic variavel that assumes 1 for clinical urgent
  patients. It's generated from \texttt{prob.urgent}.
\end{itemize}

HLA population \texttt{origin} can be defined from options:
`PT,'`API,'`AFA,'`CAU' and `HIS,' as reported for \texttt{donors\_df()}
data frame.

Defining \texttt{seed.number} allows for reproducibility.

\hypertarget{input-data}{%
\subsection{Input data}\label{input-data}}

The package presented here (\texttt{simK}), returns only fully synthetic
data because it only requires common summary statistics (that can be
found for different populations) as inputs to generate the new
microdata.

\hypertarget{bug-reports-and-contributions}{%
\subsection{Bug reports and
contributions}\label{bug-reports-and-contributions}}

Any bug reporting, feature requests, or other feedback will be welcomed
by \href{https://github.com/txopen/simK/issues}{submitting an issue} in
our repository. When reporting a bug, please ensure that a reproducible
example of your code is included so that we may respond to your issue
promptly.

\hypertarget{funding}{%
\section{Funding}\label{funding}}

This project received the ``Antonio Morais Sarmento'' research grant
from the Portuguese Society of Transplantation. This funding had no role
in: study design; software development; the writing of the report;
neither in the decision to submit the article for publication.

\hypertarget{references}{%
\section*{References}\label{references}}
\addcontentsline{toc}{section}{References}

\hypertarget{refs}{}
\begin{CSLReferences}{1}{0}
\leavevmode\vadjust pre{\hypertarget{ref-transplantr}{}}%
Asher, J. 2020. {``Transplantr: Audit and Research Functions for
Transplantation.''}
\url{https://cran.r-project.org/web/packages/transplantr/index.html}.

\leavevmode\vadjust pre{\hypertarget{ref-Chen:2019}{}}%
Chen, Junqiao, David Chun, Milesh Patel, Epson Chiang, and Jesse James.
2019. {``{The validity of synthetic clinical data: A validation study of
a leading synthetic data generator (Synthea) using clinical quality
measures}.''} \emph{BMC Medical Informatics and Decision Making} 19 (1):
1--9. \url{https://doi.org/10.1186/s12911-019-0793-0}.

\leavevmode\vadjust pre{\hypertarget{ref-ET}{}}%
EuroTransplant. 2020. {``{ETKAS and ESP}.''}
\url{https://www.eurotransplant.org/allocation/eurotransplant-manual/}.

\leavevmode\vadjust pre{\hypertarget{ref-Gonzales:2023}{}}%
Gonzales, Aldren, Guruprabha Guruswamy, and Scott R. Smith. 2023.
{``{Synthetic data in health care: A narrative review}.''} \emph{PLOS
Digital Health} 2 (1): e0000082.
\url{https://doi.org/10.1371/journal.pdig.0000082}.

\leavevmode\vadjust pre{\hypertarget{ref-ctv:2021}{}}%
Lima, B. 2021. {``A Draft for a CRAN Task View on Synthetic Data.''}
2021. \url{https://github.com/balima78/SyntheticData}.

\leavevmode\vadjust pre{\hypertarget{ref-Lima:2013a}{}}%
Lima, B., and H. Alves. 2013. {``{HLA-A, -C, -B, AND -DRB1 allele ic and
haplotypic diversity in bone marrow volunteer donors from northern
Portugal}.''} \emph{Organs, Tissues \& Cells}, no. 16: 19--26.
\url{https://12f11c1f-960a-f627-594d-b8ce276384f7.filesusr.com/ugd/3e838e_dc548dede99a4db5869c3d2c20c2d16f.pdf?index=true}.

\leavevmode\vadjust pre{\hypertarget{ref-Lima:2020}{}}%
---------. 2020. {``{Access to kidney transplantation: a bioethical
perspective}.''} \emph{Portuguese Journal of Nephrology \& Hypertension}
34 (2): 76--78. \url{https://doi.org/10.32932/pjnh.2020.07.070}.

\leavevmode\vadjust pre{\hypertarget{ref-Lima:2022b}{}}%
Lima, B., T. Henriques, and H. Alves. 2022. {``{Kidney allocation rules
simulator}.''} \emph{Transplant Immunology} 72: 101578.
\url{https://doi.org/10.1016/j.trim.2022.101578}.

\leavevmode\vadjust pre{\hypertarget{ref-Lima:2013b}{}}%
Lima, B., M. Mendes, and H. Alves. 2013. {``{Kidney Transplant
allocation in Portugal}.''} \emph{Portuguese Journal of Nephrology \&
Hypertension} 27 (4): 313--16.
\url{http://www.bbg01.com/cdn/clientes/spnefro/pjnh/46/artigo_14.pdf}.

\leavevmode\vadjust pre{\hypertarget{ref-Lima:2022a}{}}%
Lima, B., and F. Reis. 2022. {``Histoc: Histocompatibility Analysis
Performed by Kidney Allocation Systems.''} \emph{Journal of Open Source
Software} 7 (80): 4807. \url{https://doi.org/10.21105/joss.04807}.

\leavevmode\vadjust pre{\hypertarget{ref-PT}{}}%
Ministry of Health. 2007. {``Ordinance Nº 6537/2007.''}
\url{http://www.ipst.pt/files/TRANSPLANTACAO/DOACAOETRANSPLANTACAO/Despacho_6537_2007.pdf}.

\leavevmode\vadjust pre{\hypertarget{ref-UK}{}}%
National Health Service Blood and Transplant. 2017. {``{UK Transplant
Registry, Organ Donation and Transplantation}.''}
\url{http://odt.nhs.uk/uk-transplant-registry/data/}.

\leavevmode\vadjust pre{\hypertarget{ref-nmdp:2023}{}}%
NMDP. 2023. {``A-b-Drb1 224 Haplotype Frequencies.''} 2023.
\url{https://bioinformatics.bethematchclinical.org/hla-resources/haplotype-frequencies/a-b-drb1-224-haplotype-frequencies/}.

\leavevmode\vadjust pre{\hypertarget{ref-Philpott:2017}{}}%
Philpott, Don. 2017. \emph{{A guide to Federal terms and Acronyms}}.
Bernan Press.

\leavevmode\vadjust pre{\hypertarget{ref-R}{}}%
R Core Team. 2021. {``R: A Language and Environment for Statistical
Computing.''} Vienna, Austria: R Foundation for Statistical Computing.
\url{https://www.R-project.org/}.

\leavevmode\vadjust pre{\hypertarget{ref-Walonoski:2018}{}}%
Walonoski, Jason, Mark Kramer, Joseph Nichols, Andre Quina, Chris
Moesel, Dylan Hall, Carlton Duffett, Kudakwashe Dube, Thomas Gallagher,
and Scott McLachlan. 2018. {``{Synthea: An approach, method, and
software mechanism for generating synthetic patients and the synthetic
electronic health care record}.''} \emph{Journal of the American Medical
Informatics Association} 25 (3): 230--38.
\url{https://doi.org/10.1093/jamia/ocx079}.

\end{CSLReferences}

\end{document}
